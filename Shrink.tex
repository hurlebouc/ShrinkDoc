% !TEX encoding = Windows Latin 1
\documentclass[10pt]{article}

\usepackage{geometry}
\geometry{a4paper}
\usepackage{amsmath}
\usepackage{stmaryrd}
\usepackage{listings}
\usepackage{moreverb}
\usepackage{amsfonts,amssymb,verbatim}
\usepackage{amsthm}
\usepackage{mathrsfs}
\usepackage{graphicx}
\usepackage[latin1]{inputenc}
\usepackage{color}
\usepackage[T1]{fontenc}
\usepackage{lastpage}
\usepackage[francais]{babel}
\usepackage{fancyhdr}
\usepackage{calc}
\newcommand{\noind}{\noindent}
\usepackage[bookmarks,colorlinks,breaklinks]{hyperref}  % PDF hyperlinks, with coloured links
\definecolor{dullmagenta}{rgb}{0.4,0,0.4}   % #660066
\definecolor{darkblue}{rgb}{0,0,0.4}
\hypersetup{linkcolor=blue,citecolor=blue,filecolor=dullmagenta,urlcolor=darkblue} % coloured links
%\hypersetup{linkcolor=black,citecolor=black,filecolor=black,urlcolor=black} % black links, for printed output

\newcommand{\imp}[1]{\textit{\textbf{#1}} }

\definecolor{lightgrey}{gray}{0.75}
\definecolor{gris}{rgb}{0.9,0.9,0.9} 
\definecolor{bleuclair}{rgb}{0.7,0.7,1}
\lstset{ basicstyle={\ttfamily \small}, language=c, keywordstyle=\color{blue}, frame=lines, backgroundcolor=\color{gris}, breaklines = true, showstringspaces=false} 

 
\newcommand{\R}{\mathbb{R}}
\newcommand{\Q}{\mathbb{Q}}
\newcommand{\Z}{\mathbb{Z}}
\newcommand{\N}{\mathbb{N}}
\newcommand{\K}{\mathbb{K}}

\newcommand{\cons}{\mathcal{C}}
%\renewcommand{\S}{\mathcal{S}}
\newcommand{\sig}{\mathcal{S}}
\newcommand{\D}{\mathcal{D}}
\newcommand{\predicat}{\mathcal{P}}
\newcommand{\F}{\mathcal{F}}
\newcommand{\X}{\mathcal{X}}
\newcommand{\type}{\mathcal{T}}
\newcommand{\AND}{\vee}
\newcommand{\OR}{\wedge}
\newcommand{\NOT}[1]{\overline{#1}}
\newcommand{\IMPLY}{\Rightarrow}
\newcommand{\f}{\Sigma}
\renewcommand{\d}{\Delta}
\newcommand{\gom}{\texttt{Gom} }
\newcommand{\valide}{\vDash}
\newcommand{\nvalide}{\nvDash}
\newcommand{\ssi}{si et seulement si }
\newcommand{\define}{\overset{\mathrm{def}}=}
\newcommand{\dom}{\mathtt{dom}}
\newcommand{\codom}{\mathtt{codom}}
\newcommand{\term}{\tau}
\newcommand{\size}{\mathtt{size}}
\newcommand{\ta}{terme alg�brique }
\newcommand{\tas}{termes alg�briques }
\newcommand{\st}{sous-terme }
\newcommand{\sts}{sous-termes }
\newcommand{\std}{sous-terme direct }
\newcommand{\stds}{sous-termes directs }
\newcommand{\ST}{\gamma}
\newcommand{\DERIVE}{\delta}
\newcommand{\algo}{algorithme }
\newcommand{\algos}{algorithmes }
\newcommand{\Algo}{Algorithme }
\newcommand{\Algos}{Algorithmes }

















\newcommand{\qc}{Quickcheck }


\newcommand{\abs}[1]{\left| #1 \right|}
\newcommand{\norme}[1]{\left\Vert #1 \right\Vert}
\newcommand{\tribarre}[1]{\left\Vert \! \! \: \left| #1 \right| \! \!  \: \right\Vert}
\newcommand{\dint}[1]{\displaystyle{\int} \! \!  \! \! \displaystyle{\int_{ #1}} }
\newcommand{\tint}[1]{\displaystyle{\int} \! \!  \! \! \displaystyle{\int} \! \!  \! \! \displaystyle{\int_{#1}} }
\newcommand{\scal}[1]{\left\langle #1 \right\rangle}
\newcommand{\ent}[1]{\left\llbracket#1\right\rrbracket}
\newcommand{\somme}[2]{\displaystyle{\sum_{#1}^{#2}}}
\newcommand{\integrale}[2]{\displaystyle{\int_{#1}^{#2}}}
\newcommand{\e}{\mathrm{e}}
\def\jfrac#1#2{\raisebox{2pt}{$#1$}/\raisebox{-2pt}{$#2$}}   %%%%  jolie fraction



\pagestyle{fancy}
%\lhead{ \textit{QUICKCHECK} } 
  %\chead{ }
%\rhead{\textit{R�duction des contre-exemples}}

%\fancyfoot{\cfoot{\thepage}}
\renewcommand{\headrulewidth}{0.6pt}
\definecolor{mygrey}{gray}{0.75}
\renewcommand{\headrule}{{\color{lightgrey}
 \hrule width\headwidth height\headrulewidth \vskip-\headrulewidth}}
 
\theoremstyle{plain} 
\newtheorem{theo}{Th�or�me } 
\newtheorem{rappel}{Rappel du th�or�me }
\newtheorem{lemme}{Lemme }

\begin{document}

%\section{G�n�ration de termes}
%	\subsection{Vocabulaire utilis�} % !TEX root = ../shrink.tex

%	\subsection{Pr�sentation de l'algorithme} % !TEX root = ../shrink.tex

	
\section{Utilisation des termes g�n�r�s}% !TEX root = ../shrink.tex

\noind On souhaite pouvoir d�crire quel est le fonctionnement attendu du programme. On utilise pour cela les r�sultats des calculs de pr�dicats au niveau de la syntaxe. On donne un sens � une phrase logique en lui adjuvant un ou plusieurs mod�le. Tout ce qui va �tre dit dans cette section n'est qu'une mise en commun de notations. Seule la fin de la derni�re partie de cette section est utile au sens o� elle donne l'impl�mentation JAVA de la th�orie.

\subsection{Syntaxe}

\noind Il est possible d'�crire des phrases math�matique telles que 

$$\forall x \in D, \; \left( (P(x)) \Rightarrow (Q(f(x))) \right)$$

\noind Pour formaliser la syntaxe, on se muni d'un ensemble infini $\X$ de symboles de \textit{variables} et un ensemble $\sig$ de symboles de signatures. Chaque signature a une \textit{arit�}. On peut alors d�finir la notion de termes de mani�re inductive :
\begin{itemize}
\item toute variable est un terme,
\item toute signature d'arit� 0 est un terme,
\item si $s$ est un symbole de fonction d'arit� $n$, et que $t_1, t_2, \dots, t_n$ sont des termes, alors $s(t_1, t_2, \dots, t_n)$ est un terme.
\end{itemize}
L'ensembles de termes est not� $\term$.\\

\noind Enfin, on peut d�finir, �galement de mani�re inductive, l'ensemble $\F$ des formules : �tant donn� $\predicat$ un ensemble de symboles de pr�dicats et $\D$ est un ensemble de symboles de domaines, on a que
\begin{itemize}
\item tout pr�dicat d'arit� 0 est une formule,
\item si $P$ est un symbole de pr�dicat d'arit� $n$, et que $t_1, t_2, \dots, t_n$ sont des termes, alors $P(t_1, t_2, \dots, t_n)$ est une formule,
\item si $F_1$ et $F_2$ sont des formules, alors $(F_1) \AND (F_2)$, $(F_1) \OR (F_2)$ et $ (F_1) \IMPLY (F_2)$ sont des formules,
\item si $F$ est une formule, $\NOT{F}$ est une formule,
\item si $f$ est une formule et $D$ un symbole de domaine, $\forall x \in D, \; (F)$ et $\exists x \in D, \; (F)$ sont des formules.\\
\end{itemize}

\noind Cette d�finition d'une formule se traduit facilement en \gom (listing \ref{formule}). La phrase math�matique pr�c�dente sera donc sous la forme du listing \ref{f1}.

\begin{lstlisting}[caption=D�finition d'une formule en \gom, label=formule, float]
 Args = ListArgs(Term*)
 
 Term = Var(name:String)
     | Sig(name:String, args:Args)
 
 Formula = Predicate(name:String, args:Args)
        | And(f1:Formula, f2:Formula)
        | Or(f1:Formula, f2:Formula)
        | Imply(f1:Formula, f2:Formula)
        | Not(f:Formula)
        | Forall(var:String, domain:String, f:Formula)
        | Exists(var:String, domain:String, f:Formula)

\end{lstlisting}


\begin{lstlisting}[caption=Exemple d'une formule, label=f1, float=htb]
Forall("x",
	"D",
	Imply(
		Predicate(
			"P",
			ListArgs(
				Var(
					"x"
				)
			)
		)
		Predicate(
			"Q",
			ListArgs(
				Sig(
					"f",
					ListArgs(
						Var(
							"x"
						)
					)
				)
			)
		)
	)
)
\end{lstlisting}

\noind Une fois la syntaxe �crite, il faut lui donner un \textit{sens}. En effet, chaque mot qui n'est pas d�finit dans la grammaire doit l'�tre par ailleurs. Ainsi, on voudra dire par exemple que la phrase ci-dessus signifie
\begin{center}
\textit{Pour tout entier $x$, $x$ pair implique $x+1$ impair.}
\end{center}

\subsection{S�mantique}

\subsubsection{Interpr�tation d'une formule}


\noind On d�finit d'abord $\d$ un ensemble de domaines et on pose
$$\forall n \in \N, \; \f_{\d, n} \define \displaystyle{\bigcup_{E \in \d^n,F \in \d}} F^{E} \; \mathrm{et} \; \f'_{\d,n} \define \displaystyle{\bigcup_{E \in \d^n}} \{0,1\}^{E}.$$

\noind $\f_{\d,n}$ est donc l'ensemble des fonctions d'arit� $n$ dont le codomaine et le domaine sont des �l�ments de $\d$ et de $\d^n$, et $\f'_{\d, n}$ est l'ensemble des fonctions d'arit� $n$ de codomaine $\{0,1\}$ et de domaine dans $\d^{n}$.

\noind On peut interpr�ter une formule syntaxique issue d'un ensemble de signatures $\sig$, de pr�dicats $\predicat$ et de domaines $\D$ en lui associant une \textit{interpr�tation} $I$ qui est la connaissance d'un couple $(\d_I, \ent{\cdot}_I)$ o�
\begin{itemize}
\item $\d_I$ est un ensemble de domaine
\item $\ent{\cdot}_I$ est une fonction telle que 


$$ \forall s \in \sig, \; \ent{s}_I \in \f_{\d_I,n}, \; \mbox{o� $s$ est d'arit� $n$}$$
$$ \forall P \in \predicat, \; \ent{P}_I \in \f'_{\d_I, n}, \; \mbox{o� $P$ est d'arit� $n$}.$$
$$ \forall D \in \D, \; \ent{D}_I \in \d_I.$$

\end{itemize}

\noind Ainsi, si on applique ce formalisme � notre exemple, on a 
\begin{itemize}
\item $\d_I = \{\N\}$
\item $\ent{D}_I = \N$
\item $\ent{f}_I : x\in \N \rightarrow x+1 \in \N$
\item $\ent{P}_I : x \in \N \rightarrow 1$ si $x$ est pair et $0$ sinon
\item $\ent{Q}_I : x \in \N \rightarrow 0$ si $x$ est pair et $1$ sinon.\\
\end{itemize}

\noind On a ainsi appliqu� une s�mantique � notre formule math�matique.

\subsubsection{�valuation d'une formule}

\noind On va traiter ici la question de savoir comment, �tant donn�e une formule et une interpr�tation, on peut affirmer qu'une formule est \textit{valide} ou non selon l'interpr�tation, que l'on note $I \valide F$. On introduit alors la notion de \textit{valuation}. 

\noind Une valuation $\sigma$ est une fonction � domaine fini $\{x_1, x_2, \dots,x_n \} \in \X^n$ o� les $x_i$ sont deux � deux distinctes, qui associe � chaque $x_i$ un �l�ment de $E \in \d$. L'ensemble des valuations est not� $\Upsilon$. On peut alors �valuer un terme.
\begin{itemize}
\item Si $x$ est un symbole de variable, $\scal{x \vert \sigma}_I = \sigma(x)$.
\item Si $s$ est un symbole de signature d'arit� $n$ et $t_1, t_2, \dots, t_n$ sont des termes, 

$$\scal{s(t_1, t_2, \dots,t_n) \vert \sigma}_I = \ent{s}_I\left(\scal{t_1 \vert \sigma}_I, \scal{t_2 \vert \sigma}_I, \dots, \scal{t_n \vert \sigma}_I\right)$$
\end{itemize}

\noind On peut alors d�finir la validit� d'une formule selon l'interpr�tation $I$ et la valuation $\sigma$, que l'on note $I\valide_\sigma F$. Si $F, F_1, F_2$ sont des formules, $P$ est un symbole de pr�dicat d'arit� $n$ et $t_1, t_2, \dots, t_n$ sont des termes, alors
\begin{itemize}
\item $ I \valide _\sigma P(t_1, t_2, \dots, t_n) \Leftrightarrow \ent{P}_I(\scal{t_1 \vert \sigma}_I, \scal{t_2 \vert \sigma}_I, \dots, \scal{t_n \vert \sigma}_I) = 1$
\item $ I \valide_\sigma \NOT{F} \Leftrightarrow I \nvalide_\sigma F$
\item $ I \valide_\sigma (F_1) \AND (F_2) \Leftrightarrow I \valide_\sigma F_1 \;\mathrm{et}\; I \valide_\sigma F_2$
\item $ I \valide_\sigma (F_1) \OR (F_2) \Leftrightarrow I \valide_\sigma F_1 \;\mathrm{ou}\; I \valide_\sigma F_2$
\item $ I \valide_\sigma (F_1) \IMPLY (F_2) \Leftrightarrow I \nvalide_\sigma F_1 \;\mathrm{ou}\; I \valide_\sigma F_2$
\item $ I \valide_\sigma \forall x \in D, (F) \Leftrightarrow \forall a \in \ent{D}_I, I \nvalide_{\sigma_{\{x \leftarrow a\}}} F$
\item $ I \valide_\sigma \exists x \in D, (F) \Leftrightarrow \exists a \in \ent{D}_I, I \nvalide_{\sigma_{\{x \leftarrow a\}}} F$\\
\end{itemize}

\noind On conclue en d�finissant 

$$ I \valide F \Leftrightarrow \forall \sigma \in \Upsilon, \; I \valide_\sigma F.$$

\noind On sait maintenant dire qu'une formule est valide, dans une certaine interpr�tation, de mani�re \textit{empirique}. En effet, la preuve de validit� passe ici par l'�num�ration exhaustive de tous les cas possibles pour une interpr�tation donn�e. On a choisi cette m�thode de preuve, car on dispose d'un g�n�rateur de variables al�atoires. Comme de toute fa�on il est rarement possible de tester tous les cas, on part du principe qu'effectuer des tests sur des exemples suffisamment communs et repr�sentatifs suffit pour avoir une assez grande certitude que le programme que l'on test fonctionne\footnote{Cela s'apparente � une machine de \textit{Monte-Carlo} qui, renvoyant un contre-exemple prouve que le programme est faux, et sinon donne une certaine probabilit� que le programme soit juste}.\\

\noind Ainsi, sur notre exemple, on souhaite montrer que l'interpr�tation valide la formule. En premier lieu, on remarque que la phrase $I \valide_{\sigma} F$ ne d�pend pas de $\sigma$. En effet, $F$ est une formule close, c'est � dire que toutes les variables pr�sentes dans cette formule sont li�es. Pour montrer $I \valide F$, il suffit de le montrer sur la valuation dont l'ensemble de d�part est vide.

\noind D'autre part, notre formule est de la forme $\forall x \in D, (F')$ : on doit donc montrer que $(F')$
est vraie pour toute valuation rempla�ant la variable $x$ par une valeur dans $\ent{D}_I = \N$. C'est l� que va intervenir le g�n�rateur de termes al�atoires.


\subsubsection{Impl�mentation JAVA}

\noind On va ici reprendre les concepts vu dans les parties pr�c�dentes et voir comment les impl�menter. Si l'on veut produire quelque chose de modulable, on choisit un syst�me permettant � l'utilisateur de facilement adapter la th�orie � la pratique. Ainsi, le premier point est de permettre d'ajouter un sens aux symboles de la formule. Apr�s avoir �crit la formule dans le formalisme de \gom comme dans l'exemple du listing \ref{f1}, on obtient un objet de type \texttt{Formula}. Le logiciel lit ensuite cet objet et va faire correspondre � chaque symbole lu une signification. On a donc choisi d'utiliser plusieurs \texttt{Map} dont les entr�es sont les diff�rents symboles, et les sorties sont les objets porteurs de sens. 

\noind Ces derniers objets sont chacun d'un type donn� en fonction du symbole dont il porte la signification. On a proc�d� ainsi :
\begin{itemize}
\item toutes les �valuations de termes seront du type \texttt{ATerm} : 

$$ \forall t \in \term, \; \scal{t \vert \sigma}_I \mbox{est un \texttt{ATerm}}.$$

\item les symboles de pr�dicats de l'ensemble $\predicat$ correspondent � des objets impl�mentant l'interface \texttt{PredicatInterpretation}. Chacune de ces interpretations doit alors fournir une m�thode \texttt{isTrue} qui prend en param�tre une liste d'�valuations des termes que le symbole de pr�dicat $P$ a comme param�tre dans la formule, et qui renvoie \texttt{true} \ssi on d�cide que $P$ est vrai dans ce cas, et \texttt{false} sinon.

$$ \forall P \in \predicat, \; \ent{P}_I : \mathtt{ATerm}^n \rightarrow \{0,1\} \mbox{ o� $P$ est d'arit� $n$}.$$

\item les symboles de signature de l'ensemble $\sig$ correspondent � des objets impl�mentant l'interface \texttt{SignatureInterpretation}. Chacune de ces interpr�tations doit fournir une m�thode \texttt{compute} qui � une liste de termes associe la valeur souhait�e de l'interpr�tation du symbole dans ce cas.

$$ \forall s \in \sig, \; \ent{s}_I : \mathtt{ATerm}^n \rightarrow \mathtt{ATerm} \mbox{ o� $s$ est d'arit� $n$}.$$

\item les symboles de domaines de l'ensemble $\D$ correspondent � des objets impl�mentant l'interface \texttt{DomainInterpretation}. Chacune de ces interpretations doit fournir quatres m�thode :
	\begin{itemize}
	\item \texttt{includes} qui r�pond \texttt{true} \ssi l'objet pass� en param�tre appartient au domaine correspondant.
	
	$$ \forall D \in \D, \forall x \in \X, \forall \sigma \in \Upsilon, \; \scal{x | \sigma}_I \in \ent{D}_I \Leftrightarrow \mathtt{includes}( \scal{x | \sigma}_I ) = \mathtt{true}$$
	
	\item \texttt{getDepsDomains} qui donne tous les domaines dont d�pend le terme courant. Par exemple si on souhaite donner l'interpr�tation du domaine \texttt{Term} du listing \ref{formule}, la m�thode \texttt{getDepsDomains} doit renvoyer une liste comportant uniquement \texttt{Args} et \texttt{String}.	
	\item \texttt{lighten} qui prend un terme appartenant au domaine et renvoie tous les sous termes que l'on peut construire � partir de celui en param�tre en rempla�ant son constructeur par un constructeur plus \textit{petit}. Cette m�thode sera d�taill�e plus loin dans la partie traitant de la r�duction des contre-exemples.
	\item \texttt{chooseElement} qui prend un entier $n$ en param�tre et renvoie un �l�ments du domaine (toujours sous forme de \texttt{ATerm}) dont la taille est de l'ordre de $n$. C'est l� que l'on utilise la fonction de g�n�ration automatique de termes al�atoires. Cette m�thode sera employ�e lorsque l'on cherchera � d�montrer la validit� de formules dans l'interpr�tation donn�e. En effet, lorsque le testeur rencontrera une formule de la forme "$\forall x \in D, \; (F)$", il g�n�rera des termes $x$ de $D$ qu'il injectera dans la formule $F$ pour v�rifier sa validit�. On a par ailleurs 
	
	$$ \forall D\in \D, \; \mathtt{include(chooseElement(\mathit{n})) = true} $$
	
	\end{itemize}
\end{itemize}

\noind Ces renseignements donn�s, il est alors facile de tester la validit� d'une formule et utilisant la description formelle vue plus haut et en admettant la simplification du probl�me de tests sur seulement un �chantillon bien r�parti. 

\noind On peut m�me faire plus que cela. Si le processus de validation �choue sur sur une valuation, c'est � dire un contre-exemple, on s'est donn� les moyens de \textit{r�duire} ce contre-exemple pour �ventuellement r�soudre le bug plus facilement.

















	
\section{R�duction des contre-exemples}
	% !TEX root = ../shrink.tex

\noind On traite ici une mani�re de r�duire les termes ayant fait �chou� la validation d'une formule. La m�thode que l'on va employer ici concerne plus particuli�rement les termes \textit{alg�briques}. Nous allons commencer par poser quelques notations\footnote{Ces notations sont ind�pendantes de celle de la section pr�c�dente.}.

\subsection{Notations}
	% !TEX root = ../shrink.tex

\noind On d�fini une \textit{signature} $s \in \sig $ comme le quadruplet $(\type, \cons, \dom, \codom)$ o�
\begin{itemize}
\item $\type$ est un ensemble de \textit{types},
\item $\cons$ est un ensemble de \textit{constructeurs},
\item $\dom : \cons \rightarrow \displaystyle{\bigcup_{n \in \N}} \type^n$ donne le \textit{domaine} d'un constructeur,
\item $\codom : \cons \rightarrow \type$ donne le \textit{codomaine} d'un constructeur.\\
\end{itemize}

\noind L'\textit{arit�} d'un constructeur $c \in \cons$ est l'entier $n$ tel que $\dom(c) \in \type^n$. On peut maintenant d�finir par induction ce qu'est un \textit{\ta}.
\begin{itemize}
\item Tout constructeur d'arit� nulle est un terme.
\item Pour tout constructeur $c \in \cons$ d'arit� $n$ et les \tas $t_1, t_2, \dots, t_n$, alors $c(t_1, t_2, \dots, t_n)$ est un terme.
\end{itemize}
L'ensemble des \tas sur la signature $s$ est not� $\term_s$. Si $t = c(t1_, t_2, \dots, t_n)$, on dira que $c$ est le constructeur de $t$ et que $t_1, t_2, \dots, t_n$ sont ses \textit{\stds}et que $t_i$ est le $i$-�me \std\!\!. De m�me, un \textit{\st\!} de $t$ est soit un \std de $t$ soit un \st d'un des \stds de $t$. L'ensemble des \sts de $t$ est not� $\ST(t)$. Enfin, le \textit{chemin} $\omega_{t,t'}$ d'un \ta $t$ � l'un de ses \sts est une suite finie $u_1u_2\dots u_n \in \N^\N$ permettant de d�finir une suite de \tas $t_0t_1\dots t_n$ telle que 

$$ 
	\left\{
		\begin{array}{c}
			t_0 = t \\
			t_n = t' \\
			\forall k \in \ent{1,n}, \; t_k \mbox{ est le $u_k$-�me \st de $t_{k-1}$}
		\end{array}
	\right.
$$
\noind De par la structure d'arbre d'un terme, tout chemin entre deux \tas est unique. On peut donc d�finir la distance entre un \ta et l'un de ses \st comme la longueur du chemin les s�parant. On utilisera �galement la concat�nation de suites finies : si $\omega$ est une suite finie et $a$ une suite, alors $\omega a$ est la suites concat�n�e de $\omega$ et de $a$.\\

\noind De plus, on dira qu'un \ta $t$ \textit{d�rive} d'un type $T\in \type$ \ssi le constructeur de $t$ a $T$ pour codomaine. L'ensemble des \ta d�rivant d'un type $T$ est not� $\DERIVE(T)$. On �tendra cette notation aux termes : pour tout $t \in \term_s, \; \DERIVE(t)$ est l'ensembles des \ta d�rivant du m�me type que $t$.\\

\noind On pose �galement que la \textit{taille} d'un \ta est le nombre de constructeurs le composant, ou encore 

$$\forall t = c(t_1, t_2, \dots, t_n) \in \term_s, \; \size(t) = 1 + \sum_{i=1}^n \size(t_i). $$


\noind Ainsi, � partir d'un \ta $t \in \term_s$ faisant �chouer la validation de la formule, on cherche � construire un \ta $t'$ de plus petite taille continuant de faire �chouer la validation.

















\subsection{R�duction par \sts de m�me type}
	% !TEX root = ../shrink.tex

\noind Une premi�re m�thode \texttt{s1} permettant de r�duire un \ta $t$ consiste � remplacer $t$ par l'un de ses \sts d�rivant du m�me type. On proc�de pour cela en plusieurs �tapes. 
\begin{enumerate}
\item On r�cup�re tous les \sts d�rivant du m�me type qui minimise la distance entre eux et la racine. L'ensemble $A_t$ de ces termes est

$$ A_t = \{ t' \in \ST(t) \cap \DERIVE(t) \; \vert \; \nexists t'' \neq t', \exists a \in \N^\N,  \omega_{t, t'} = \omega_{t, t''}a\}$$ 
\item On filtre $A_t$ en supprimant tous les \tas qui ne sont pas des contre-exemples. Si le r�sultat est l'ensemble vide, on lui ajoute $t$. Finalement, on obtient $f(A_t)$.\\
\end{enumerate}

\noind On a donc cr�� une fonction \texttt{s1\_aux} qui � un \ta $t$ associe $f(A_t)$. On construit alors \texttt{s1} en it�rant \texttt{s1\_aux} sur chacun des termes de \texttt{s1\_aux}$(t)$ jusqu'� convergence.

\begin{enumerate}
	\item initialiser $E \define {t}$;
	\item initialiser un nouvel ensemble vide $E'$.
	\item pour tout �l�ment $t'$ dans $E$
	\begin{enumerate}
		\item ajouter le r�sultat de \texttt{s1\_aux}$(t')$ � $E'$.
	\end{enumerate}
	\item si $E = E'$
	\begin{enumerate}
		\item alors on s'arr�te
		\item sinon on pose $E \define E'$ et on revient � 3
	\end{enumerate}
\end{enumerate}

\noind La convergence de cet \algo ne fait pas de doute : Pour s'en convaincre, si $\left(E_n\right)_{n\in \N}$ est la suite des $E$ de l\algo\!, on peut consid�rer la diff�rence sym�trique $ \delta_n = E_n \triangle E_{n+1}$. L'\algo est effectu� de telle sorte que d�s que un �l�ment est pr�sent dans $E_n$ et dans $E_{n+1}$ (il ne fait donc plus partie de $\delta_n$), alors il reste pr�sent pour tout $n' > n$. On peut donc se contenter de r�it�re \texttt{s1\_aux} sur les �l�ments de $\delta_n \cap E_{n+1}$. On se rend alors compte que la taille maximal des �l�ments de $\delta_n$ d�croit strictement entre deux $\delta_n$ non vides successifs, ce qui impose qu'il existe un $n$ tel que $\delta_n$ soit vide, c'est � dire que $E_n = E_{n+1}$.














\subsection{R�duction du constructeur}
	% !TEX root = ../shrink.tex

\noind Le second \algo \texttt{s2} est ind�pendant de \texttt{s1} mais cherche toujours � diminuer la taille de l'�l�ment de d�part. Cette fois ci, on raisonne uniquement au niveau de la racine du terme $t$. On cherche tous les constructeur du type dont d�rive $t$ qui sont \textit{inclus} dans le constructeur de $t$. Par \textit{inclus}, on veut dire que, si $n_{c,T}$ est le nombre d'argument du constructeur $c$ de type $T$, 

$$ \forall c_1, c_2 \in \cons, \; \mbox{$c_1$ est inclus dans $c_2$} \Leftrightarrow \forall T \in \type, \; n_{c_1, T} \leqslant n_{c_2, T}.$$

\noind Pour chaque constructeur $c$ inclus dans le constructeur de $t$, on fait alterner les diff�rentes combinaisons possibles en se limitant � la r�gle que chaque \std de $t$ soit au moins une fois pr�sent dans les arguments d'un terme construit avec $c$.

%\noind Une fois la liste des contre-exemples potentiels �tablie, on la filtre de la m�me mani�re que \texttt{s1} pour supprimer les termes qui ne sont pas des contres-exemples. Si, apr�s filtrage, la liste des termes est vide, on retourne le terme initial.

\noind L'\algo \texttt{s2} est ex�cut� apr�s \texttt{s1}. La raison est que l'\algo \texttt{s2} est moins destructeur que le premier, � tel point qu'appliquer \texttt{s2} puis \texttt{s1} reviendrait � appliquer seulement \texttt{s1} et, au pire, ajouter les �l�ments obtenus en appliquant \texttt{s2} sur $t$ (en non plus sur les �l�ments de \texttt{s1}). En effet, l'algorithme \texttt{s1} cherche dans les diff�rentes branches tous les \sts de m�me type que la racine. Or \texttt{s2} ne modifie par les branches. L'application de \texttt{s2} avant celle de \texttt{s1} ne changerait que tr�s peu son r�sultat.
\subsection{Parcours du terme}
	% !TEX root = ../shrink.tex

\noind Les deux \algos pr�c�dents ne r�duisent le contre-exemple qu'en consid�ration de son constructeur. C'est � dire qu'on ne profite de la r�duction que sur un seul type (celui du terme � r�duire). Pour r�duire �galement les autres types, on choisit de r�-appliquer les deux \algos sur les termes obtenus apr�s un premier passage, mais � des \textit{profondeurs}\footnote{La \textit{profondeur} d'un \st $t'$ d'un terme $t$ est la distance les s�parant.} diff�rentes. Cependant, le filtrage doit lui rester avec le m�me type (un filtrage sur un autre type n'aurait de toute fa�on pas de sens, car le filtrage est propre � un type).

\noind Concr�tement, on ajoute aux deux \algos \texttt{s1} et \texttt{s2} la notion de profondeur : si $t = c(t_1, t_2, \dots, t_n)$, alors on d�finit \texttt{s\textit{i}Depth} de la mani�re suivante :

$$ \forall d \in \N, \; \mathtt{s\mathit{i}Depth} (t,d) = 
	\left\{\begin{array}{cc}
		\mathtt{s}i(t) & \mbox{si $d = 0$} \\
		\displaystyle{\bigcup_{k \in \ent{1,n}}} \{c(t_1, \dots, t'_k, \dots, t_n)\}_{t'_k \in \mathtt{s\mathit{i}Depth(t_k,d-1)}}& \mbox{si $d \neq 0$}
	\end{array}\right.
$$



\subsection{Filtrage}\label{filtrage}
	% !TEX root = ../shrink.tex


\noind Le filtrage $f$ permet de supprimer tous les \tas qui ne sont pas des contre-exemples. Si le r�sultat est l'ensemble vide, on lui ajoute l'�l�ment initial. Finalement, on obtient

$$ \forall t \in \term, \forall d \in \N \; f(\mathtt{s\mathit{i}Depth}(t, d)) = 
	\left\{\begin{array}{cc}
		f(\mathtt{s\mathit{i}Depth}(t, d)) & \mbox{si $f(\mathtt{s\mathit{i}Depth}(t, d)) \neq \emptyset $} \\
		\{t\}& \mbox{sinon.}
	\end{array}\right.
$$

\noind Pass�e cette fonction de filtrage, on se rend compte que l'on peut am�liorer la fonction \texttt{s1} en continuant la descente de l'arbre pour trouver les \sts de m�me type que les \tas trouv�s par l'\algo\!\!. On construit alors \texttt{s1}$'$ en it�rant $f(\mathtt{s1Depth}(\cdot, d))$ sur chacun des termes de $f(\mathtt{s1Depth}(t, d))$ jusqu'� convergence.

\begin{enumerate}
	\item initialiser $E \define {t}$;
	\item initialiser un nouvel ensemble vide $E'$.
	\item pour tout �l�ment $t'$ dans $E$
	\begin{enumerate}
		\item ajouter le r�sultat de $f(\mathtt{s1Depth}(t', d))$ � $E'$.
	\end{enumerate}
	\item si $E = E'$
	\begin{enumerate}
		\item alors on s'arr�te
		\item sinon on pose $E \define E'$ et on revient � 2
	\end{enumerate}
\end{enumerate}

\noind La convergence de cet \algo ne fait pas de doute : Pour s'en convaincre, si $\left(E_n\right)_{n\in \N}$ est la suite des $E$ de l'\algo\!, on peut consid�rer la diff�rence sym�trique $ \delta_n = E_n \triangle E_{n+1}$. L'\algo est effectu� de telle sorte que d�s que un �l�ment est pr�sent dans $E_n$ et dans $E_{n+1}$ (il ne fait donc plus partie de $\delta_n$), alors il reste pr�sent pour tout $n' > n$. On peut donc se contenter de r�it�rer \texttt{s1\_aux} sur les �l�ments de $\delta_n \cap E_{n+1}$. On se rend alors compte que la taille maximale des �l�ments de $\delta_n$ d�croit strictement entre deux $\delta_n$ non vides successifs, ce qui impose qu'il existe un $n$ tel que $\delta_n$ soit vide, c'est � dire que $E_n = E_{n+1}$.\\

\noind En ce qui concerne $f \circ \mathtt{s2Depth}(\cdot, d)$, on n'a pas besoin de r�it�rer l'\algo car on travaille exclusivement sur les constructeurs du type de la racine : $$ \mathtt{s2}' = f \circ \mathtt{s2Depth}.$$







\subsection{\Algo global}
	% !TEX root = ../shrink.tex

\noind On peut maintenant exprimer l'algorithme de r�duction des contre-exemples dans son ensemble. Si $t$ est un contre-exemple :

\begin{enumerate}
\item on pose $d$ la profondeur de $t$.
\item on pose $E \define \{t\}$.
\item pour $i$ variant de $0$ � $d$
	\begin{enumerate}
	\item on pose $E' \define \emptyset$.
	\item pour tout �l�ment $t'$ dans $E$.
		\begin{enumerate}
		\item on ajoute $\mathtt{s1}'(t', i)$ � $E'$.
		\end{enumerate}
	\item on pose $E'' \define \emptyset$.
	\item pour tout �l�ment $t'$ dans $E'$.
		\begin{enumerate}
		\item on ajoute $\mathtt{s2}'(t', i)$ � $E''$.
		\end{enumerate}
	\item $E \define E''$
	\end{enumerate}
\item on retourne le \ta de $E$ de taille minimale.
\end{enumerate}



	
	


\end{document}  




































