% !TEX TS-program = pdflatexmk
% !TEX encoding = Windows Latin 1
\documentclass[10pt]{article}

\usepackage{geometry}
\geometry{a4paper}
\usepackage{amsmath}
\usepackage{stmaryrd}
\usepackage{listings}
\usepackage{moreverb}
\usepackage{amsfonts,amssymb,verbatim}
\usepackage{amsthm}
\usepackage{mathrsfs}
\usepackage{graphicx}
\usepackage[latin1]{inputenc}
\usepackage{color}
\usepackage[T1]{fontenc}
\usepackage{lastpage}
\usepackage[francais]{babel}
\usepackage{fancyhdr}
\usepackage{calc}
\newcommand{\noind}{\noindent}
\usepackage[bookmarks,colorlinks,breaklinks]{hyperref}  % PDF hyperlinks, with coloured links
\definecolor{dullmagenta}{rgb}{0.4,0,0.4}   % #660066
\definecolor{darkblue}{rgb}{0,0,0.4}
\hypersetup{linkcolor=blue,citecolor=blue,filecolor=dullmagenta,urlcolor=darkblue} % coloured links
%\hypersetup{linkcolor=black,citecolor=black,filecolor=black,urlcolor=black} % black links, for printed output

\newcommand{\imp}[1]{\textit{\textbf{#1}} }

\definecolor{lightgrey}{gray}{0.75}
\definecolor{gris}{rgb}{0.9,0.9,0.9} 
\definecolor{bleuclair}{rgb}{0.7,0.7,1}
\lstset{ basicstyle={\ttfamily \small}, language=c, keywordstyle=\color{blue}, frame=lines, backgroundcolor=\color{gris}, breaklines = true, showstringspaces=false} 

 
\newcommand{\R}{\mathbb{R}}
\newcommand{\C}{\texttt{C}}
\newcommand{\Q}{\mathbb{Q}}
\newcommand{\Z}{\mathbb{Z}}
\newcommand{\N}{\mathbb{N}}
\newcommand{\K}{\mathbb{K}}

\renewcommand{\S}{\mathcal{S}}
\newcommand{\D}{\mathcal{D}}
\renewcommand{\P}{\mathcal{P}}
\newcommand{\F}{\mathcal{F}}
\newcommand{\X}{\mathcal{X}}
\newcommand{\T}{\mathcal{T}}
\renewcommand{\and}{\vee}
\renewcommand{\or}{\wedge}
\renewcommand{\not}[1]{\overline{#1}}
\newcommand{\imply}{\Rightarrow}
\newcommand{\f}{\Sigma}
\renewcommand{\d}{\Delta}
\newcommand{\gom}{\texttt{Gom} }





\newcommand{\qc}{Quickcheck }


\newcommand{\abs}[1]{\left| #1 \right|}
\newcommand{\norme}[1]{\left\Vert #1 \right\Vert}
\newcommand{\tribarre}[1]{\left\Vert \! \! \: \left| #1 \right| \! \!  \: \right\Vert}
\newcommand{\dint}[1]{\displaystyle{\int} \! \!  \! \! \displaystyle{\int_{ #1}} }
\newcommand{\tint}[1]{\displaystyle{\int} \! \!  \! \! \displaystyle{\int} \! \!  \! \! \displaystyle{\int_{#1}} }
\newcommand{\scal}[1]{\left\langle #1 \right\rangle}
\newcommand{\ent}[1]{\left\llbracket#1\right\rrbracket}
\newcommand{\somme}[2]{\displaystyle{\sum_{#1}^{#2}}}
\newcommand{\integrale}[2]{\displaystyle{\int_{#1}^{#2}}}
\newcommand{\e}{\mathrm{e}}
\def\jfrac#1#2{\raisebox{2pt}{$#1$}/\raisebox{-2pt}{$#2$}}   %%%%  jolie fraction



\pagestyle{fancy}
\lhead{ \textit{QUICKCHECK} } 
  %\chead{ }
\rhead{\textit{R�duction des contre-exemples}}

%\fancyfoot{\cfoot{\thepage}}
\renewcommand{\headrulewidth}{0.6pt}
\definecolor{mygrey}{gray}{0.75}
\renewcommand{\headrule}{{\color{lightgrey}
 \hrule width\headwidth height\headrulewidth \vskip-\headrulewidth}}
 
\theoremstyle{plain} 
\newtheorem{theo}{Th�or�me } 
\newtheorem{rappel}{Rappel du th�or�me }
\newtheorem{lemme}{Lemme }

\begin{document}

\section{G�n�ration de termes}
	\subsection{Vocabulaire utilis�} % !TEX root = ../shrink.tex

	\subsection{Pr�sentation de l'algorithme} % !TEX root = ../shrink.tex

	\subsection{Utilisation des termes g�n�r�s dans un contexte de  v�rification de programme}% !TEX root = ../shrink.tex

\noind On souhaite pouvoir d�crire quel est le fonctionnement attendu du programme. On utilise pour cela les r�sultats des calculs de pr�dicats au niveau de la syntaxe. On donne un sens � une phrase logique en lui adjuvant un ou plusieurs mod�le.

\subsubsection{Syntaxe}

\noind Il est possible d'�crire des phrases math�matique telles que 

$$\forall x \in D, \; \left( P(x) \Rightarrow Q(f(x)) \right)$$

\noind On utilise pour cela la syntaxe du listing \ref{formule}.

\begin{lstlisting}[caption=D�finition d'une formule, label=formule]
 Args = ListArgs(Term*)
 
 Term = Var(name:String)
     | Sig(name:String, args:Args)
 
 Formula = Predicate(name:String, args:Args)
        | And(f1:Formula, f2:Formula)
        | Or(f1:Formula, f2:Formula)
        | Imply(f1:Formula, f2:Formula)
        | Not(f:Formula)
        | Forall(var:String, domain:String, f:Formula)
        | Exists(var:String, domain:String, f:Formula)

\end{lstlisting}

\noind La phrase math�matique pr�c�dente sera donc sous la forme du listing \ref{f1}.

\begin{lstlisting}[caption=Exemple d'une formule, label=f1, float=htb]
Forall("x",
	"D",
	Imply(
		Predicate(
			"P",
			ListArgs(
				Var(
					"x"
				)
			)
		)
		Predicate(
			"Q",
			ListArgs(
				Sig(
					"f",
					ListArgs(
						Var(
							"x"
						)
					)
				)
			)
		)
	)
)
\end{lstlisting}

\noind On peut ainsi essayer de d�crire le fonctionnement d'un programme. Une fois la syntaxe �crite, il faut lui donner un sens. En effet, chaque mot qui n'est pas d�finit dans la grammaire doit l'�tre par ailleurs. Ainsi, on voudra dire par exemple que la phrase ci-dessus signifie
\begin{center}
\textit{Pour tout entier $x$, $x$ pair implique $x+1$ impair.}
\end{center}

\subsubsection{Semantique}




	
\section{R�duction des contre-exemples}

	
	


\end{document}  




































