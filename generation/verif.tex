% !TEX root = ../shrink.tex

\noind On souhaite pouvoir d�crire quel est le fonctionnement attendu du programme. On utilise pour cela les r�sultats des calculs de pr�dicats au niveau de la syntaxe. On donne un sens � une phrase logique en lui adjuvant un ou plusieurs mod�le.

\subsubsection{Syntaxe}

\noind Il est possible d'�crire des phrases math�matique telles que 

$$\forall x \in D, \; \left( P(x) \Rightarrow Q(f(x)) \right)$$

\noind On utilise pour cela la syntaxe du listing \ref{formule}.

\begin{lstlisting}[caption=D�finition d'une formule, label=formule]
 Args = ListArgs(Term*)
 
 Term = Var(name:String)
     | Sig(name:String, args:Args)
 
 Formula = Predicate(name:String, args:Args)
        | And(f1:Formula, f2:Formula)
        | Or(f1:Formula, f2:Formula)
        | Imply(f1:Formula, f2:Formula)
        | Not(f:Formula)
        | Forall(var:String, domain:String, f:Formula)
        | Exists(var:String, domain:String, f:Formula)

\end{lstlisting}

\noind La phrase math�matique pr�c�dente sera donc sous la forme du listing \ref{f1}.

\begin{lstlisting}[caption=Exemple d'une formule, label=f1, float=htb]
Forall("x",
	"D",
	Imply(
		Predicate(
			"P",
			ListArgs(
				Var(
					"x"
				)
			)
		)
		Predicate(
			"Q",
			ListArgs(
				Sig(
					"f",
					ListArgs(
						Var(
							"x"
						)
					)
				)
			)
		)
	)
)
\end{lstlisting}

\noind On peut ainsi essayer de d�crire le fonctionnement d'un programme. Une fois la syntaxe �crite, il faut lui donner un sens. En effet, chaque mot qui n'est pas d�finit dans la grammaire doit l'�tre par ailleurs. Ainsi, on voudra dire par exemple que la phrase ci-dessus signifie
\begin{center}
\textit{Pour tout entier $x$, $x$ pair implique $x+1$ impair.}
\end{center}

\subsubsection{Semantique}



