% !TEX root = ../shrink.tex

\noind On d�fini une \textit{signature} $s \in \S $ comme le quadruplet $(\T, \C, \dom, \codom)$ o�
\begin{itemize}
\item $\T$ est un ensemble de \textit{types},
\item $\C$ est un ensemble de \textit{constructeurs},
\item $\dom : \C \rightarrow \displaystyle{\bigcup_{n \in \N}} \T^n$ donne le \textit{domaine} d'un constructeur,
\item $\codom : \C \rightarrow \T$ donne le \textit{codomaine} d'un constructeur.\\
\end{itemize}

\noind L'\textit{arit�} d'un constructeur $c \in \C$ est l'entier $n$ tel que $\dom(c) \in \T^n$. On peut maintenant d�finir par induction ce qu'est un \textit{\ta}.
\begin{itemize}
\item Tout constructeur d'arit� nulle est un terme.
\item Pour tout constructeur $c \in \C$ d'arit� $n$ et les \tas $t_1, t_2, \dots, t_n$, alors $c(t_1, t_2, \dots, t_n)$ est un terme.
\end{itemize}
L'ensemble des \tas sur la signature $s$ est not� $\t_s$. Si $t = c(t1_, t_2, \dots, t_n)$, on dira que $c$ est le constructeur de $t$ et que $t_1, t_2, \dots, t_n$ sont ses \textit{\stds}\!\!. De m�me, un \textit{\st\!} de $t$ est soit un \std de $t$ soit un \st d'un des \stds de $t$.

\noind De plus, on dira qu'un \ta $t$ \textit{d�rive} d'un type $T\in \T$ \ssi le constructeur de $t$ a $T$ pour codomaine. Enfin, la taille d'un \ta est le nombre de constructeurs le composant, ou encore 

$$\forall t = c(t_1, t_2, \dots, t_n) \in \t_s, \; \size(t) = 1 + \sum_{i=1}^n \size(t_i). $$

\noind Ainsi, � partir d'un \ta $t \in \t_s$ faisant �chouer la validation de la formule, on cherche � construire un \ta $t'$ de plus petite taille continuant de faire �chouer la validation.
















