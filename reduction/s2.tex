% !TEX root = ../shrink.tex

\noind Le second \algo \texttt{s2} est ind�pendant de \texttt{s1} mais cherche toujours � diminuer la taille de l'�l�ment de d�part. Cette fois ci, on raisonne uniquement au niveau de la racine du terme $t$. On cherche tous les constructeur du type dont d�rive $t$ qui sont \textit{inclus} dans le constructeur de $t$. Par \textit{inclus}, on veut dire que le, si $n_{c,T}$ est le nombre d'argument du constructeur $c$ de type $T$, 

$$ \forall c_1, c_2 \in \cons, \; \mbox{$c_1$ est inclus dans $c_2$} \Leftrightarrow \forall T \in \type, \; n_{c_1, T} \leqslant n_{c_2, T}.$$

\noind Pour chaque constructeur $c$ inclus dans le constructeur de $t$, on fait alterner les diff�rentes combinaisons possibles en se limitant � la r�gle que chaque \std de $t$ soit au moins une fois pr�sent dans les arguments d'un terme construit avec $c$.

%\noind Une fois la liste des contre-exemples potentiels �tablie, on la filtre de la m�me mani�re que \texttt{s1} pour supprimer les termes qui ne sont pas des contres-exemples. Si, apr�s filtrage, la liste des termes est vide, on retourne le terme initial.

\noind L'\algo \texttt{s2} est ex�cut� apr�s \texttt{s1}. La raison est que l'\algo \texttt{s2} est moins destructeur que le premier, a tel point que appliquer \texttt{s2} puis \texttt{s1} reviendrait � appliquer seulement \texttt{s1} et, au pire, ajouter les �l�ments obtenu en appliquant \texttt{s2} sur $t$ (en non plus sur les �l�ments de \texttt{s1}). En effet, l'algorithme \texttt{s1} chercher dans les diff�rentes branches tous les \sts de m�me type que la racine. Or \texttt{s2} ne modifie par les branches. L'application de \texttt{s2} avant celle de \texttt{s1} ne changerait que tr�s peu son r�sultat.